\subsection*{Gitin käyttö}

Master-\/haara pidetään siistinä ja siellä aina toimiva/kääntyvä ohjelmaversio.

Development-\/haarassa ohjelman kehitysversio ja uudet toiminnot luodaan developmentin päältä tehtyihin työhaaroihin.

Uuden haaran luonti lokaalisti komennolla {\ttfamily git checkout -\/b $<$branch$>$} Uuden haaran vienti remoteen {\ttfamily git push -\/u $<$remote$>$ $<$branch$>$}

Vastaavasti haaran poisto lokaalisti tapahtuu komennolla {\ttfamily git branch -\/d $<$branch$>$} Ja haaran poisto remotesta {\ttfamily git push -\/d $<$remote$>$ $<$branch$>$}

Työhaaran vaihto {\ttfamily git checkout $<$branch$>$}

Pidetään commit-\/viestit lyhyinä ja ytimekkäinä. Käytetään commiteissa käskymuotoa. E.\-g.\char`\"{}\-Fix a bug x\char`\"{}.

Haarojen yhdistäminen\-: Mene haaraan johon toinen haara yhdistetään {\ttfamily git checkout $<$main branch$>$} Yhdistä toinen haara tähän haaraan mergellä {\ttfamily git merge $<$feature branch$>$}

Jos fast-\/forward ei onnistu (tulee merge konflikteja), korjaa konfliktit manuaalisesti. Korjausten jälkeen committaa muutokset esim viestillä \char`\"{}\-Fix merge conflicts\char`\"{} ja suorita merge. 